\startcomponent c-base-type
\product prd-glib-note

\chapter{基本类型}

出于易用与可移植性等方面的考虑,GLib 定义了一些常用的数据类型。

\section{标准 C 所缺乏的数据类型}

事实上,GLib 所定义的这部分数据类型,多数在 C99 中已有定义,但是它们均非标准 C 的内建类型,通常是在各种 C 标准库的头文件定义(并且不同的 C 标准库所提供的头文件名可能会不相同)。GLib 之所以提供相应的替代类型,主要是将 C 标准库隐匿,向用户提供一组风格较为一致的数据类型。

C 语言没有提供布尔类型,GLib 基于 int 类型定义了 gboolean 类型,并定义了 FALSE 与 TRUE 宏:
\starttyping[option=c]
#ifndef FALSE
#define FALSE (0)
#endif

#ifndef TRUE
#define TRUE (!FALSE)
#endif
\stoptyping

为了增强 C 程序的可移植性,C 语言标准提供了 size\_t 类型,一种无符号整型变量,适于计量存储空间中可容纳的数据项目个数,在数组下标和内存管理函数之类的地方广泛使用,例如 sizeof 操作符的返回结果即为 size\_t 类型。size\_t 类型的存储宽度是平台依赖的,例如在 32 位环境中,其宽度为 32 位,而在 64 位环境中,其宽度则为 64 位。size\_t 的有符号变体是 ssize\_t。GLib 定义了 gsize 与 gssize 类型,分别等同于 size\_t 与 ssize\_t。GLib 的这种克隆行为显得有点多余,但事实上,size\_t 与 ssize\_t 一般是在 C 标准库的头文件中定义,程序中使用时通常需要:

\starttyping[option=c]
#include <stdlib.h>
\stoptyping

\noindent 为此 GLib 不辞劳苦,让我们有了一点小解脱。gsize 与 gssize 的取值范围分别为 [0, G\_MAXSIZE] 与 [G\_MINSSIZE, G\_MAXSSIZE]。

C99 中提供了 off64\_t 类型,是 64 位宽的整型类型,用于表示文件偏移量,主要用于大文件的数据存取。像对待 size\_t 那样,GLib 也提供了 off64\_t 的克隆,即 goffset 类型,并定义了相应取值范围 [G\_MINOFFSET, G\_MAXOFFSET]。

很多情况下,C 程序员会将指针类型的变量强制转换为整型变量进行一些运算(多用于需要精确控制数据在内存中的精确布局)。在 32 位平台上,由于指针类型的宽度和 int 相同,所以指针类型与 int 类型的转换问题通常不受重视,但是在 64 位平台上,由于目前几乎所有的 64 位操作系统均采用 LP64 模型,即 int 的宽度是 32 位,而指针类型的宽度为 64 位,此时指针类型与 int 类型的转换便会存在隐患。为此,C99 中提供了 intptr\_t 数据类型,以保证 C 程序跨平台移植的安全性。intptr\_t 是有符号的整型类型,其无符号变体为 uintptr\_t,GLib 提供的替代类型分别为 gintptr 与 guintptr,并且提供了相应的输入/输出格式 G\_GINTPTR\_MODIFIER 与 G\_GINTPTR\_FORMAT,例如:

\starttyping[option=c]
#include <glib.h>
... ...
gchar *s = "hello world!";
gintptr p = (gintptr)s;
g_print ("%#" G_GINTPTR_MODIFIER "x\n", p);
g_print ("%" G_GINTPTR_FORMAT "\n", p);
... ...
\stoptyping

记住,对于使用 GLib 库的程序,gcc 的编译命令为:

\starttyping[numbering=no]
$ gcc `pkg-config --cflags --libs glit-2.0` foo.c -o foo
\stoptyping

\section{跨平台等宽整型类型}

这部分数据类型是一组固定宽度的整型类型,有 g[u]int[8, 16, 32, 64]\footnote{我喜欢的简写方式,可解读为:gint8, guint8, gint16, guint16, gint32, guint32, gint64, guint64。},GLib 所做的工作就是在它支持的所有平台上保证它们的宽度一致。

设 $x \in \{8, 16, 32, 64\}$,gint$x$ 的取值范围为 [G\_MININT$x$, G\_MAXINT$x$],\crlf
guint$x$ 的取值范围为 [0, G\_MAXUINT$x$];g[u]int$x$ 的输入/输出格式为 G\_GINT$x$\_[MODIFIER, FORMAT]。

\section{标准 C 数据类型封装}

这部分数据类型均为标准 C 的内建类型,GLib 采用封装的形式进行了些许简化。

首先,C 的无类型指针 void * 与指向常量数据的无类型指针 const void * 分别被 GLib 定义为 gpointer 与 gconstpointer。

然后,C 的无符号字符、短整型、整型与长整型,即 unsigned [char, short, int, long] 被 GLib 定义为 gu[char, short, int, long]。

\section{还有一些画蛇添足的类型}

为了保持与前三部分数据类型编码风格的一致,GLib 行为艺术般的将标准 C 内建的数据类型 $\{x | x \in$ {char, int, short, long, float, double} 重新取名为 g$x$,并且界定它们的取值范围为 [-G\_MAX$X$, G\_MAX$X$],$X \in$ {CHAR, INT, SHORT, LONG, FLOAT, DOUBLE}。

\section{数值宏}

GLib 对数学计算中经常用到的几个数值提供了宏定义。有些 C 标准库提供了相关宏定义,但也有未提供的。GLib 能做到的就是它可以保证它会一直提供这些宏。

G\_E 表示自然基底 $e$。

G\_PI, G\_PI\_2, G\_PI\_4 分别表示 $\pi, 2\pi, 4\pi$。

G\_LN2 与 G\_LN10 分别表示 $\ln 2$ 与 $\ln 10$。

G\_LOG\_2\_BASE\_10 表示 \asciimath{log_{10}2} 。

G\_SQRT2 表示 $\sqrt{2}$。

此外,GLib 还提供了 IEEE754 单/双精度浮点数移码的宏定义,并提供便于访问 IEEE754 格式单/双精度浮点数的符号、阶码与尾数的 union 类型,例如:

\starttyping[option=c]
GFloatIEEE754 a = {3.1415};
g_print ("%u\n", a.mpn.sign);
g_print ("%u\n", a.mpn.biased_exponent);
g_print ("%u\n", a.mpn.mantissa);
\stoptyping

\section{想知道更多}

GLib 基本类型的详细说明见 GLib 参考手册的“GLib Fundamentals”部分以及 glib/gtypes.h 文件。

GLib 的基本类型的外延部分还提供了整型类型与指针类型的原子操作,由于我的程序环境不涉及多线程编程问题,所以此部分内容被我故意忽略了,若需要了解这部分内容,可查阅 GLib 参考手册的“Atomic Operations”部分。

\stopcomponent
