%%%%%%%%%%%%%%%%%%%%%%%%%%%%%%%%%%%%%%%%%%%%%%%%%%%%%%%%%%%%%%%%%
% Contents: Who contributed to this Document
% $Id: contrib.tex,v 1.2 2003/03/19 20:57:44 oetiker Exp $
%%%%%%%%%%%%%%%%%%%%%%%%%%%%%%%%%%%%%%%%%%%%%%%%%%%%%%%%%%%%%%%%%
%中文~4.20~翻译:zpxing@bbs.ctex  email: zpxing at gmail dot com
%%%%%%%%%%%%%%%%%%%%%%%%%%%%%%%%%%%%%%%%%%%%%%%%%%%%%%%%%%%%%%%%%

%\chapter{Thank you!}
\chapter{致谢!}
%\noindent Much of the material used in this introduction comes from
%an Austrian introduction to \LaTeX\ 2.09 written in German by:
\noindent 在这份介绍中使用的许多材料来自一个奥地利人使用德语撰写
的~\LaTeX\ 2.09 介绍:
\begin{verse}
\contrib{Hubert Partl}{partl@mail.boku.ac.at}%
{Zentraler Informatikdienst der Universit\"at f\"ur Bodenkultur Wien}
\contrib{Irene Hyna}{Irene.Hyna@bmwf.ac.at}%
   {Bundesministerium f\"ur Wissenschaft und Forschung Wien}
\contrib{Elisabeth Schlegl}{no email}%
   {in Graz}
\end{verse}

%If you are interested in the German document, you can find a version
%updated for \LaTeXe{} by J\"org Knappen at\\
%\CTAN|info/lshort/german|
如果你对德文文档有兴趣,有一个由 J\"org Knappen 针对 \LaTeXe{} 更新的版本,在 CTAN 的位置是:
\texttt{\CTAN|info/lshort/german|}

%\newpage \noindent The
%following individuals helped with corrections, suggestions and
%material to improve this paper. They put in a big effort to help me
%get this document into its present shape. I would like to
%sincerely thank all of them. Naturally, all the mistakes you'll find
%in this book are mine. If you ever find a word that is spelled
%correctly, it must have been one of the people below dropping me a
%line.
\newpage
\noindent 下列人士为改进此文提供了校正、建议和素材。
他们的不懈努力帮助我把这份文档实现为现在这样子。我对他们所有人
表示诚挚的感谢。当然,你在本书中找到的所有错误都是我的失误。而
你见到的每一个拼写正确的单词,都一定是由于下面列出的这些人之一通知了我。

{ \flushleft\small
Rosemary Bailey,        %r.a.bailey@qmw.ac.uk 0.2
Marc Bevand,            % <bevand_m@epita.fr>
Friedemann Brauer,      %fbrauer@is.dal.ca 3.4
Jan Busa,               % <busaj@ccsun.tuke.sk>
Markus Br\"uhwiler,     % <m.br@switzerland.org>
Pietro Braione,         % <braione@elet.polimi.it>
David Carlisle,         %GONE carlisle@cs.man.ac.uk 1.0
Jos\'e Carlos Santos,   % <jcsantos@fc.up.pt>
Neil Carter,            % N.Carter@Swansea.ac.uk
Mike Chapman,           %chapman@eeh.ee.ethz.ch 3.16
Pierre Chardaire,       % <pc@sys.uea.ac.uk
Christopher Chin,       %chris.chin@rmit.edu.au 3.1
Carl Cerecke,           %cdc@cosc.canterbury.ac.nz>
Chris McCormack,        %GONE chrismc@eecs.umich.edu 0.1
Wim van Dam,            %GONE wimvdam@cs.kun.nl 2.2
Jan Dittberner,         %jan@jan-dittberner.de 3.15
Michael John Downes,    %<mjd@ams.org> 14 Oct 1999
Matthias Dreier,        %dreier@ostium.ch
David Dureisseix,       %dureisse@lmt.ens-cachan.fr 1.1
Elliot,                 %GONE enh-a@minster.york.ac.uk 1.1
Hans Ehrbar,            %ehrbar@econ.utah.edu
Daniel Flipo,           %Daniel.Flipo@univ-lille1.fr
David Frey,             %david@eos.lugs.ch 2.2
Hans Fugal,             %hans@fugal.net
Robin Fairbairns,       %Robin.Fairbairns@cl.cam.ac.uk 0.2 1.0
J\"org Fischer,        %j.fischer@xpoint.at 3.16
Erik Frisk,             %frisk@isy.liu.se 3.4
Mic Milic Frederickx,   % <mic.milic@web.de>
Frank,                  %frank@freezone.co.uk 11 Feb 2000
Kasper B. Graversen,    % <kbg@dkik.dk>
Arlo Griffiths,         % <A.Griffiths@let.leidenuniv.nl>
Alexandre Guimond,      %guimond@IRO.UMontreal.CA 0.9
Andy Goth,              % <unununium@openverse.com>
Cyril Goutte,           %goutte@ei.dtu.dk 2.1 2.2
Greg Gamble,            %gregg@maths.uwa.edu.au 2.2
Frank Fischli,          % <fischlifaenger@gmx.ch>
Morten H{\o}gholm,     % <moho01ab@student.cbs.dk>
Neil Hammond,           %nfh@dmu.ac.uk 0.3
Rasmus Borup Hansen,    %GONE rbhfamos@math.ku.dk 0.2 0.9 0.91 0.92 1.9.9
Joseph Hilferty,        % <hilferty@fil.ub.es>
Bj\"orn Hvittfeldt,     %bjorn@hvittfeldt.com 3.13
Martien Hulsen,         %M.A.Hulsen@WbMt.TUDelft.NL 1.0 1.1
Werner Icking,          %<Werner.Icking@gmd.de> 3.1
Jakob,                  %diness@get2net.dk
Eric Jacoboni,          %GONE jacoboni@enseeiht.fr 0.1 0.9
Alan Jeffrey,           %alanje@cogs.sussex.ac.uk 0.2
Byron Jones,            %bj@dmu.ac.uk 1.1
David Jones,            %GONE djones@CA.McMaster.dcss.insight 1.1
Johannes-Maria Kaltenbach, %<kaltenbach@zeiss.de> 3.01
Michael Koundouros,     % <mkoundouros@hotmail.com>
Andrzej Kawalec,        %GONE akawalec@prz.rzeszow.pl 1.9.9
Sander de Kievit,       %Skievit@ucu.uu.nl
Alain Kessi,            %ALAIN_KESSI@HOTMAIL.COM 2.2
Christian Kern,         %ck@unixen.hrz.uni-oldenburg.de 2.1
Tobias Klauser,     %tklauser@access.unizh.ch 4.17
J\"org Knappen,         %knappen@vkpmzd.kph.uni-mainz.de 0.1
Kjetil Kjernsmo,        %<kjetil.kjernsmo@astro.uio.no> 3.2
Maik Lehradt,           %greek@uni-paderborn.de 0.1
R\'emi Letot,           % <r_letot@yahoo.com>
Flori Lambrechts,       % <f.lambrechts@softhome.net>
Axel Liljencrantz,  % <Axel.Liljencrantz@byv.kth.se>
Johan Lundberg,         %p99jlu@physto.se
Alexander Mai,          %Alexander.Mai@physik.tu-darmstadt.de 3.8
Hendrik Maryns,         %hendrik.maryns@ugent.be
Martin Maechler,        %<maechler@stat.math.ethz.ch> 2.2
Aleksandar S Milosevic, % <aleksandar.milosevic@yale.edu>
Henrik Mitsch,          % <Henrik.Mitsch@gmx.at>
Claus Malten,           %GONE <ASI138%BITNET.DJUKFA11@BITNET.CEARN> 1.1
Kevin Van Maren,        % <vanmaren@fast.cs.utah.edu>  24 Nov 1999
Richard Nagy,           % r.nagy@nameshield.net
Philipp Nagele,         % Philipp.Nagele@t-systems.com
Lenimar Nunes de Andrade, % <lenimar@mat.ufpb.br> Fri, 12 Nov 1999
Manuel Oetiker,         % manuel@oetiker.ch
Urs Oswald,             % osurs@bluewin.ch
Martin Pfister,     % m@rtinpfister.ch
Demerson Andre Polli,   % polli@linux.ime.usp.br
Nikos Pothitos,     % <n.pothitos@di.uoa.gr>
Maksym Polyakov         % <polyama@myrealbox.com>
Hubert Partl,           %partl@mail.boku.ac.at 0.2 1.1
John Refling,           %refling@sierra.lbl.gov 0.1 0.9
Mike Ressler,           %ressler@cougar.jpl.nasa.gov 0.1 0.2 0.9 1.0 1.9.9
Brian Ripley,           %ripley@stats.ox.ac.uk 2.1
Young U. Ryu,           %ryoung@utdallas.edu 2.1
Bernd Rosenlecher,      %9rosenle@informatik.uni-hamburg.de 10 Feb 2000
Chris Rowley,           %C.A.Rowley@open.ac.uk 0.91
Risto Saarelma,         %risto.saarelma@cs.helsinki.fi
Hanspeter Schmid,       %schmid@isi.ee.ethz.ch
Craig Schlenter,        %cschle@lucy.ee.und.ac.za 0.1 0.2 0.9
Gilles Schintgen,       %gschintgen@internet.lu
Baron Schwartz,         % <bps7j@cs.virginia.edu>
Christopher Sawtell,    %<csawtell@xtra.co.nz> 1 Sep 1999
Miles Spielberg,        %zeibach@hotmail.com
Geoffrey Swindale,      % <geofftswin@ntlworld.com>
Laszlo Szathmary,       % <szathml@delfin.klte.hu>
Boris Tobotras,         % <tobotras@jet.msk.su>
Josef Tkadlec,          %tkadlec@math.feld.cvut.cz 2.0 2.2
Scott Veirs,            %scottv@ocean.washington.edu
Didier Verna,           %verna@inf.enst.fr 2.2
Fabian Wernli,          %wernli@iap.fr 3.2
Carl-Gustav Werner,     % <Carl-Gustav.Werner@math.lu.se> 11 Oct 1999,3.16
David Woodhouse,        % <dwmw2@infradead.org> 3.16
Chris York,             % <c.s.york@Cummins.com>  21 Nov 1999
Fritz Zaucker,          %zaucker@ee.ethz.ch 3.0
Rick Zaccone,           %zaccone@bucknell.edu 2.2
and Mikhail Zotov.      %zotov@eas.npi.msu.su 3.1

}

\vspace*{\stretch{1}}

\pagebreak

\begin{center}
\Large  4.20 中文版致謝!
\end{center}

中文 \TeX{} 學會啓動的 lshort-zh-cn 修正計劃已經完工!
本項計劃歷時八個月,參加的朋友有:

\begin{center}
\begin{tabular}{ll}
\hline
\textbf{C\TeX 論壇 ID}  & \textbf{執行章節}  \\
\hline
zpxing    &   前言、第二章、第五章 1-2.4 {\&} 3、第六章 \\
Frogge    &   第一章  \\
liwenjun  &   第三章  \\
lijian605 &   第四章  \\
gprsnl    &   第五章 2.5-2.11 \\
\hline
\end{tabular}
\end{center}

haginile 和 Frogge 通讀了全篇,并寫出了詳盡的勘誤表。
blackold 對于第二章亦有所貢獻。最后由 zpxing 統籌全書。

\noindent\dotfill

\begin{center}
\Large 原 3.20 中文版致謝!
\end{center}

本文档的翻译工作由 C\TeX{} 版主“经典问题”倡议,历经近十个月才得以完成。
期间参与翻译工作的朋友有:

\begin{center}
\begin{tabular}{lll}
\hline
\textbf{C\TeX 论坛 ID}  & \textbf{翻译章节}  & \textbf{源文件名} \\
\hline
经典问题    &   前言    &   overview.tex  \\
高原之狼    &   第一章  &   things.tex   \\
controlong  &   第二章  &   typeset.tex  \\
cxterm      &   第三章  &   math.tex, lssym.tex \\
aloft       &   第四章  &   spec.tex   \\
ganzhi      &   第五章  &   custom.tex \\
\hline
\end{tabular}
\end{center}

\vspace{20pt}

\begin{flushright}
\textbf{在此特向这些奉献者表示感谢!}
\end{flushright}

\endinput
%%% Local Variables:
%%% mode: latex
%%% TeX-master: "lshort"
%%% End:
