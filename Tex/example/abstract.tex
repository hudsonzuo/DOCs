%%&=& &=& &=& &=& &=& &=& &=& &=& &=& &=& &=& &=& &=& &=& &=& &=& &=& &=& &=& &=& &=& &=& &=& &=& 
%% Filename: abstract.tex
%% Encoding: UTF-8
%% Author: Tux Yuen - yxshuai@gmail.com
%% Created: 2011-12-06 12:32
%% Last modified: 2011-12-11 13:39
%%&=& &=& &=& &=& &=& &=& &=& &=& &=& &=& &=& &=& &=& &=& &=& &=& &=& &=& &=& &=& &=& &=& &=& &=& 
\documentclass[a4paper]{article}
\begin{document}
\begin{abstract}

	论文的摘要是对论文研究内容和成果的高度概括。摘要应对论文所研究的问题及其研究目
	的进行描述,对研究方法和过程进行简单介绍,对研究成果和所得结论进行概括。摘要应
	具有独立性和自明性,其内容应包含与论文全文同等量的主要信息。使读者即使不阅读全
	文,通过摘要就能了解论文的总体内容和主要成果。

	论文摘要的书写应力求精确、简明。切忌写成对论文书写内容进行提要的形式,尤其要避
	免“第 1 章……;第 2 章……;……”这种或类似的陈述方式。

	本文介绍郑州大学论文模板的来由及使用方法,并对模板所涉及的宏包及自定义命令进行
	简要的说明。

	本文的创新点主要有:
	\begin{itemize}
		\item 用例子来解释模板的使用方法;
		\item 用废话来填充无关紧要的部分;
		\item 一边学习摸索一边编写新代码。
	\end{itemize}

	关键词是为了便于做文献索引和检索工作而从论文中选取出来用以表示全文主题内容信息的
	单词或术语。关键词的应体现论文特色,具有语义性,在论文中有明确的出处,并应尽量采
	用《汉语主题词表》或各专业主题词表提供的规范词。在摘要下方另起一行注明,一般3~8
	个,之间用空格分开。为便于国际交流,应标注与中文对应的英文关键词。

	\keywords{\TeX{\quad{ \LaTeX{\quad{ \XeLaTeX{\quad{ 模板 \quad{ 论文}
\end{abstract}

\begin{englishabstract}
	
	An abstract of a dissertation is a summary and extraction of research work
	and contributions. Included in an abstract should be description of research
	topic and research objective, brief introduction to methodology and research
	process, and summarization of conclusion and contributions of the
	research. An abstract should be characterized by independence and clarity and
	carry identical information with the dissertation. It should be such that the
	general idea and major contributions of the dissertation are conveyed without
	reading the dissertation.
	
	An abstract should be concise and to the point. It is a misunderstanding to
	make an abstract an outline of the dissertation and words ``the first
	chapter'', ``the second chapter'' and the like should be avoided in the
	abstract.
	
	Key words are terms used in a dissertation for indexing, reflecting core
	information of the dissertation. An abstract may contain a maximum of 5 key
	words, with semi-colons used in between to separate one another.
	
	\englishkeywords{\TeX{\quad{ \LaTeX{\quad{ \XeLaTeX{\quad{ template\quad{ thesis}
\end{englishabstract}

}<++>}<++>}<++>}<++>}<++>}<++>}<++>}<++>}<++>}<++>}<++>}<++>}<++>}<++>
\end{document}
