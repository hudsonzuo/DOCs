\documentclass[12pt,a4paper]{article}
\usepackage{fontspec,xunicode,xltxtra}
\usepackage{titlesec}
\usepackage[top=1in,bottom=1in,left=1.25in,right=1.25in]{geometry}

\titleformat{\section}{\Large\xbsong}{\thesection}{1em}{}

\XeTeXlinebreaklocale ``zh''
\XeTeXlinebreakskip = 0pt plus 1pt minus 0.1pt

%\newfontfamily\song{Simsun (Founder Extended)}
%\newfontfamily\bwei{FZBeiWeiKaiShu-S19S}
%\newfontfamily\zbhei{FZZhanBiHei-M22T}
\newfontfamily\xzt{FZXiaoZhuanTi-S13T}
\newfontfamily\xbsong{FZXiaoBiaoSong-B05}
\newfontfamily\dbsong{FZDaBiaoSong-B06}
\newfontfamily\gulif{FZGuLi-S12T}
\newfontfamily\gulij{FZGuLi-S12S}
\newfontfamily\kai{FZKai-Z03}
\newfontfamily\hei{FZHei-B01}
\newfontfamily\whei{WenQuanYi Zen Hei}
\newfontfamily\fsong{FZFangSong-Z02}
\newfontfamily\lanting{FZLanTingSong}
\newfontfamily\boya{FZBoYaSong}
\newfontfamily\lishu{FZLiShu-S01}
\newfontfamily\lishuII{FZLiShu II-S06}
\newfontfamily\yao{FZYaoTi-M06}
\newfontfamily\zyuan{FZZhunYuan-M02}
\newfontfamily\xhei{FZXiHei I-Z08}
\newfontfamily\xkai{FZXingKai-S04}
\newfontfamily\ssong{FZShuSong-Z01}
\newfontfamily\bsong{FZBaoSong-Z04}
\newfontfamily\nbsong{FZNew BaoSong-Z12}
\newfontfamily\caiyun{FZCaiYun-M09}
\newfontfamily\hanj{FZHanJian-R-GB}
\newfontfamily\songI{FZSongYi-Z13}
\newfontfamily\hcao{FZHuangCao-S09}
\newfontfamily\wbei{FZWeiBei-S03}
\newfontfamily\huali{FZHuaLi-M14}
\setmainfont{FZLanTingSong}

\renewcommand{\baselinestretch}{1.25}

\begin{document}

\title{\whei XeTeX使用小结}
\author{\fsong 何勃亮}
\date{\kai2009年6月21日}

\maketitle

\section{简介}
以前使用CJK进行中文的排版,需要自己生成字体库,近日,出现了XeTeX,可以比较好的解决中文字体问题,不需要额外
生成LaTeX字体库,直接使用计算机系统里的字体。

\section{字体列表}
本文使用了大量本机自带的字体。


\begin{table}[htbp]
	\caption{字体列表}
	
	\centering
	\begin{tabular}{|l|c|r|}
		\hline
		\hei 字体 & \hei 命令 & \hei 字体效果 \\
		\hline
		\kai 宋体 & \verb+\song+ & \song 宋体 \\
		\kai 楷体 & \verb+\kai+ & \kai 楷体 \\
		\kai 黑体 & \verb+\hei+ & \hei 黑体 \\
		\kai 仿宋体 & \verb+\fsong+ & \fsong 仿宋体 \\
		\kai 文泉驿黑体 & \verb+\whei+ & \whei 文泉驿黑体 \\
		\kai 书宋体 & \verb+\ssong+ & \ssong 书宋体 \\
		\kai 报宋体 & \verb+\bsong+ & \bsong 报宋体 \\
		\kai 新报宋体 & \verb+\nbsong+ & \nbsong 新报宋体 \\
		\kai 兰亭宋体 & \verb+\lanting+ & \lanting 兰亭宋体 \\
		\kai 博雅宋体 & \verb+\boya+ & \boya 博雅宋体 \\
		\kai 宋体一 & \verb+\songI+ & \songI 宋体一 \\
		\kai 隶书 & \verb+\lishu+ & \lishu 隶书 \\
		\kai 隶书二 & \verb+\lishuII+ & \lishuII 隶书二 \\
		\kai 古隶简体 & \verb+\gulij+ & \gulij 古隶简体 \\
		\kai 古隶繁体 & \verb+\gulif+ & \gulif 古隶繁体 \\
		\kai 华隶书 & \verb+\huali+ & \huali 华隶书 \\
		\kai 小标宋 & \verb+\xbsong+ & \xbsong 小标宋 \\
		\kai 大标宋 & \verb+\dbsong+ & \dbsong 大标宋 \\
		\kai 小篆体 & \verb+\xzt+ & \xzt 小篆体 \\
		\kai 姚体 & \verb+\yao+ & \yao 姚体 \\
		\kai 准圆 & \verb+\zyuan+ & \zyuan 准圆 \\
		\kai 细黑一 & \verb+\xhei+ & \xhei 细黑一 \\
		\kai 行楷书 & \verb+\xkai+ & \xkai 行楷书 \\
		\kai 彩云体 & \verb+\caiyun+ & \caiyun 彩云体 \\
		\kai 汉简书 & \verb+\hanj+ & \hanj 汉简书 \\
		\kai 魏碑体 & \verb+\wbei+ & \wbei 魏碑体 \\
		\hline
	\end{tabular}
\end{table}

\end{document}

