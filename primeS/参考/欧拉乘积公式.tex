%\documentclass[a4paper,11pt]{article}
\documentclass[a4paper]{article}
\usepackage{xunicode}
\usepackage{fontspec}
\usepackage{xltxtra}
\usepackage{indentfirst}
\usepackage{amsfonts}
\usepackage{amsmath}
\usepackage{abstract}
\usepackage{xeCJK}                   % 中英文混排
\usepackage{bm}
\usepackage[numbers,sort&compress]{natbib}
\renewcommand{\citet}[1]{\textsuperscript{\cite{#1}}}
\renewcommand{\citep}[1]{\textsuperscript{\cite{#1}}}
\addtolength{\bibsep}{-0.5 em} % 缩小参考文献间的垂直间距
\setlength{\bibhang}{2em}
\renewcommand{\bibfont}{\normalfont}% 不宜使用中文字号,因为中文字号定义中包含了多余的行距

%\usepackage{amssymb}
%\usepackage{color}
%\usepackage{CJK}
%\usepackage{fancybox}
\defaultfontfeatures{Mapping=tex-text}  % 启用tex风格字符
\setmainfont[BoldFont=文泉驿正黑]{AR PL SungtiL GB}

\begin{document}
广义 Euler 乘积公式: 设 f(n) 为满足 $f(n_1)f(n_2) = f(n_1n_2)$, 且$ \sum_n|f(n)| < ∞ $ 的函数 ($n_1$、 $n_2$ 均为自然数), 则:
$\sum_n f(n) = \prod_p [1+f(p)+f(p2)+f(p3)+ \ldots]$

Euler 本人的证明: 除了上述证明方法外, Euler 原始论文中的证明方法也相当简洁, 值得介绍一下。 仍以广义 Euler 乘积公式为框架, 注意到——利用 f(n) 的性质:

$f(2)\sum_n f(n) = f(2)+f(4)+f(6)+ \cdots$

因此:

$[1-f(2)]\sum_n f(n) = f(1)+f(3)+f(5)+ \cdots$

上式右端的一个显著特点, 是所有含有因子 2 的 f(n) 项都消去了 (这种逐项对消有赖于 $\sum_n |f(n)| < ∞$, 即 $\sum_n f(n)$ 绝对收敛这一条件)。 类似地, 以 [1-f(3)] 乘以上式, 则右端所有含有因子 3 的 f(n) 项也将被消去, 依此类推, 以所有 [1-f(p)] (p 为素数) 乘以上式, 右端便只剩下了 f(1), 即:

$\prod_p [1-f(p)]\sum_n f(n) = f(1) = 1$

其中最后一步再次使用了 f(n) 的性质, 即 $(1)f(n)=f(n) ⇒ f(1)=1$ 将上式中的无穷乘积移到等式右边, 显然就得到了广义 Euler 乘积公式。
\end{document}
